
\documentclass{article}
\usepackage{amssymb, amsmath, amsfonts, latexsym}
\usepackage[total={18cm, 25cm}, centering]{geometry}
 \usepackage{graphicx} % figuras
\newcommand{\R}{\mathbb{R}}
\newcommand{\N}{\mathbb{N}}
\newcommand{\Z}{\mathbb{Z}}
\newcommand{\Q}{\mathbb{Q}}
\newcommand{\C}{\mathbb{C}}

%\usepackage{cancel}


\begin{document}
Algunas veces se quiere factorizar una matriz imponiendo condiciones sobre los factores. Una condición habitual es
la de factores no negativos y dado que los vectores no negativos no constituyen un espacio vectorial, entonces no se
pueden aplicar los principios del álgebra lineal. Es allí donde la perspectiva de la optimización es importante y permite
crear síntesis como:\\
El SVD truncado prevee la mejor aproximación en términos de error cuadrático en el contexto de las matrices de rango $k$.\\
Aunque el problema de factorización se enfrenta como un problema de optimización, se verá que una de las soluciones óptimas
involucra factores cuyas columnas presentan el fenómenos de ortogonalidad.\\

\textbf{Formulación}
Si las columnas de $V$ tienen la cualidad de ortonormalidad, entonces la representación reducidad de los datos $D$ es
obtenda como $U=DV$. El problema de optimización se formula, tratando de maximizar la energía de $U=DV$.\\

\begin{equation*}
\begin{aligned}
& \underset{V _{d\times k} }{\text{minimize}}
& & \|DV\| _{F} ^{ 2}   \\
& \text{sujeto a}
& & V ^{T}V = I _{k }
\end{aligned}
\end{equation*}

Si $\overline{V} _{r} $ es la r-ésima columna de $V$, note que
$$\|DV\| _{F} ^{ 2} = \displaystyle \sum _{r=1} ^{ k}  \|D \overline{V} _{r}\| _{F} ^{ 2}
= \sum _{r=1} ^{ k} \overline{V} _{r}^{T} D ^{T}D \overline{V} _{r}   $$

\end{document}
